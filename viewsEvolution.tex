\documentclass[12pt]{paper}
\usepackage[english]{babel}
\usepackage{indentfirst}
\usepackage{graphicx}
\usepackage{latexsym}
\usepackage{amsmath}
\usepackage{amsthm}
\usepackage{amssymb,amsfonts}
\usepackage{multicol}
\usepackage{xcolor}
\usepackage{changepage}
\usepackage{hyperref}
\usepackage{fancyhdr}
% \usepackage{showkeys}
\usepackage{textcomp}
\usepackage{mathtools}
\usepackage{comment}
\usepackage[normalem]{ulem}
\usepackage{marginnote}
\usepackage[numbers,sort&compress]{natbib}
%%%%%
\mathtoolsset{showonlyrefs=false}

\newcommand{\ga}{\alpha}
\newcommand{\gb}{\beta}
\newcommand{\gam}{\gamma}
\newcommand{\gd}{\delta}
\newcommand{\eps}{\epsilon}
\newcommand{\veps}{\varepsilon}
\newcommand{\gz}{\zeta}
\newcommand{\gt}{\theta}
\newcommand{\gi}{\iota}
\newcommand{\gk}{\kappa}
\newcommand{\gl}{\lambda}
\newcommand{\gs}{\sigma}
\newcommand{\go}{\omega}
\newcommand{\Gam}{\Gamma}
\newcommand{\gD}{\Delta}
\newcommand{\gT}{\Theta}
\newcommand{\gL}{\Lambda}
\newcommand{\gS}{\Sigma}
\newcommand{\gO}{\Omega}

%%%%%%%%%

\newcommand{\pt}[1]{\left( #1\right)}
\newcommand{\pq}[1]{\left[ #1 \right]}
\newcommand{\pg}[1]{\left\{ #1\right\}}
\newcommand{\figref}[1]{\figurename~\ref{#1}}
\newcommand{\red}[1]{\textcolor{red}{#1}}
\newcommand{\blue}[1]{\textcolor{blue}{#1}}
\newcommand{\gray}[1]{\textcolor{gray}{#1}}
% \newcommand{\wikilink}[1]\href{plan.pdf}{Plan}\,(\href{plan.tex}{tex})
% \renewcommand{\baselinestretch}{2} 
%%%%%%%%%
\newcommand{\wikilink}[2] { \href{#1.pdf}{#2}\,(\href{#1.tex}{edit})}

\setlength{\textheight}{20cm}
%\changetext{0cm}{0cm}{0cm}{0cm}{0cm}
\changepage{3.4cm}{5cm}{-2cm}{-2.5cm}{-2.0cm}{-2cm}{0.3cm}{-0.5cm}{0.1cm}
%{length of the text}{width of the text}{}{shift to the right}
%{}{}{}{}{from text to pagenumber}
%\pagenumbering{roman}
\pagestyle{fancy}
\lhead{\bf \today}
\chead{\bf Views on evolution}
\rhead{EG}
\title{Views on evolution -- \today}
\author{EG}
\date{\today}

\begin{document}
 \maketitle
 
 \wikilink{home}{Home}

\wikilink{viewsEvolution}{Self}
 
 \tableofcontents
 \section{Phenotype to Genotype Relation}
\subsection{Phenotype=Genotype and origin of life}

Clearly we can see from the working example\cite{Cliff1993} that adaptive evolution is possible in 
the case when $G=P$. It is not clear, however, if any kind of ``infinite growth of complexity'' is 
possible in such systems. It's very likely, it's not. It is also not clear what ``complexity'' is. 
Nevertheless, we can say that despite parameter space can be very big for $G=P$ it is very simple. 

When it comes to RNA evolution, in principle we can say that if $G=P$ doesn't forbid adaptive 
evolution and von Neumann replicators can function and evolve\cite{Freitas2004}\textbf{(??)} 
without separation of blueprint and machine then \textbf{RNA on its own is capable of adaptive 
evolution}.

In RNA case:
\begin{itemize}
 \item Genotype -- RNA sequence
 \item Phenotype -- RNA fold or, catalytic ability or something else. For the fold it seems, 
thought, we can build function $G\to P$, which is not necessarily reversible: $G\to\{P_i\}$ 
\item Fitness function \textbf{(both bad, both possibly tautological. I would say the tautology 
problem is strong here, the following are more of product of selection than function itself.)}:
  \subitem - Number of RNA copies.
  \subitem - Number of vesicles  with this type of RNA.
\item The system can adapt to:
  \subitem - Skew in monomer distributions
  \subitem - Fluctuation in food supply
  \subitem - New chemicals arriving
\end{itemize}


 \subsection{The system}
 
 An entity: it has a phenotype and genotype and some kind of fitness function. It can be the case, 
when $G=P$ or $G\neq P$
 \paragraph{Example systems:}
\begin{enumerate}
 \item \textbf{Set of vectors.} Vectors are of fixed length and their elements belong to some 
alphabet, for example $\{0,1\}$
  \subitem Genotype -- sequence, Phenotype -- sequence $G=P$ obviously. Fitness function -- 
closeness to some vector by some metric.
  \item \textbf{Protein $G=P$.} A sequence of amino acids folds. Different folds are considered as 
different organisms
  \subitem Genotype -- set of dihedral angles, Phenotype = Genotype. Fitness function -- Free energy
  \item \textbf{Protein $G\neq P$.} A sequence of amino acids folds. Different folds are considered 
as different organisms
  \subitem Genotype -- set of dihedral angles. Phenotype = fitness function = Free energy
  \item \textbf{Robots in a maze.} Robots of certain fixed construction and varying neural networks.
  \subitem Genotype -- Sequence of weights of neural networks. Phenotype -- behavior of the robot. 
Fitness function -- how close the robot will come to a target. It can measure distance from 
itself to the target. $G\neq P$
\end{enumerate}

 
 
 \section{Optimization methods and their relation to biological and artificial evolution}
 
 \subsection{Gradient descent}
 
 \paragraph{Statement.} 

\section{Types of Memory}
\begin{enumerate}
 \item Solution memory
 \item Linage memory
 \item Lifetime memory
\end{enumerate}


\section{Antifragility}
\subsection{Conditional of survival}
Ancient cells/polymers/metabolism systems had to survive and better benefit from:
\begin{itemize}
 \item Variation in food supply (as total)
 \item Variation in ratios of different food types
 \item Mutations at every level of informational flow
 \item New arriving chemicals
 \item Variation in temperature
 \item Movement around??
\end{itemize}
Lineage memory implemented as lifetime memory gives resilience, does it give antifragility? 
Systems with solution only memory are very fragile \textbf{(are they?)}, because they gradient 
descent quickly \textbf{(do they?)}.

\paragraph{Copying of I.} A process of copying $I\to I \to I$ through generations can happen with 
and without mistakes:
\begin{eqnarray}
  I_1\to I_1 \to I_1\\
  I_1\to I_1 \to I_2
\end{eqnarray}
Given that mutations are present now and were even more frequent event in the past(no 
error-fixing mechanism), the system would be better of if it \textbf{benefits} from mutations to a 
certain degree. Let's say $I_1$ gives a phenotype $P_1$, which is good for condition $C_1$ and 
$I_2\to P_2@C_2$. Then if the correct mutation $I_1\to I_2$ had happened before change $C_1\to 
C_2$ happened then our organism will survive. If we have many organisms, then some will survive, 
however it is a sort of resilience. If we get to keep both $I_1,\,I_2$ in the system, we would be 
better of -- we loose none at the change. How can we keep both? What will prevent loss of $I$, 
which is not used currently?



\bibliographystyle{apalike}
\bibliography{/data/research/31.mendeleyBibtex/library}
\end{document}
