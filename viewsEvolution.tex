\documentclass[12pt]{paper}
\usepackage[english]{babel}
\usepackage{indentfirst}
\usepackage{graphicx}
\usepackage{latexsym}
\usepackage{amsmath}
\usepackage{amsthm}
\usepackage{amssymb,amsfonts}
\usepackage{multicol}
\usepackage{xcolor}
\usepackage{changepage}
\usepackage{hyperref}
\usepackage{fancyhdr}
% \usepackage{showkeys}
\usepackage{textcomp}
\usepackage{mathtools}
\usepackage{comment}
\usepackage[normalem]{ulem}
\usepackage{marginnote}
\usepackage[numbers,sort&compress]{natbib}
%%%%%
\mathtoolsset{showonlyrefs=false}

\newcommand{\ga}{\alpha}
\newcommand{\gb}{\beta}
\newcommand{\gam}{\gamma}
\newcommand{\gd}{\delta}
\newcommand{\eps}{\epsilon}
\newcommand{\veps}{\varepsilon}
\newcommand{\gz}{\zeta}
\newcommand{\gt}{\theta}
\newcommand{\gi}{\iota}
\newcommand{\gk}{\kappa}
\newcommand{\gl}{\lambda}
\newcommand{\gs}{\sigma}
\newcommand{\go}{\omega}
\newcommand{\Gam}{\Gamma}
\newcommand{\gD}{\Delta}
\newcommand{\gT}{\Theta}
\newcommand{\gL}{\Lambda}
\newcommand{\gS}{\Sigma}
\newcommand{\gO}{\Omega}

%%%%%%%%%

\newcommand{\pt}[1]{\left( #1\right)}
\newcommand{\pq}[1]{\left[ #1 \right]}
\newcommand{\pg}[1]{\left\{ #1\right\}}
\newcommand{\figref}[1]{\figurename~\ref{#1}}
\newcommand{\red}[1]{\textcolor{red}{#1}}
\newcommand{\blue}[1]{\textcolor{blue}{#1}}
\newcommand{\gray}[1]{\textcolor{gray}{#1}}
% \newcommand{\wikilink}[1]\href{plan.pdf}{Plan}\,(\href{plan.tex}{tex})
% \renewcommand{\baselinestretch}{2} 
%%%%%%%%%
\newcommand{\wikilink}[2] { \href{#1.pdf}{#2}\,(\href{#1.tex}{edit})}

\setlength{\textheight}{20cm}
%\changetext{0cm}{0cm}{0cm}{0cm}{0cm}
\changepage{3.4cm}{5cm}{-2cm}{-2.5cm}{-2.0cm}{-2cm}{0.3cm}{-0.5cm}{0.1cm}
%{length of the text}{width of the text}{}{shift to the right}
%{}{}{}{}{from text to pagenumber}
%\pagenumbering{roman}
\pagestyle{fancy}
\lhead{\bf \today}
\chead{\bf Views on evolution}
\rhead{EG}
\title{Views on evolution -- \today}
\author{EG}
\date{\today}

\begin{document}
 \maketitle
 
 \wikilink{home}{Home}

\wikilink{viewsEvolution}{Self}
 
 \tableofcontents
 \section{The system}
 
 An entity: it has a phenotype and genotype and some kind of fitness function. It can be the case, 
when $G=P$ or $G\neq P$

 \paragraph{Example systems:}
\begin{enumerate}
 \item \textbf{Set of vectors.} Vectors are of fixed length and their elements belong to some 
alphabet, for example $\{0,1\}$
  \subitem Genotype -- sequence, Phenotype -- sequence $G=P$ obviously. Fitness function -- 
closeness to some vector by some metric.
  \item \textbf{Protein $G=P$.} A sequence of amino acids folds. Different folds are considered as 
different organisms
  \subitem Genotype -- set of dihedral angles, Phenotype = Genotype. Fitness function -- Free energy
  \item \textbf{Protein $G\neq P$.} A sequence of amino acids folds. Different folds are considered 
as different organisms
  \subitem Genotype -- set of dihedral angles. Phenotype = fitness function = Free energy
  \item \textbf{Robots in a maze.} Robots of certain fixed construction and varying neural networks.
  \subitem Genotype -- Sequence of weights of neural networks. Phenotype -- behavior of the robot. 
Fitness function -- how close the robot will come to a target. It can measure distance from 
itself to the target. $G\neq P$
\end{enumerate}

 
 
 \section{Optimization methods and their relation to biological and artificial evolution}
 
 \subsection{Gradient descent}
 
 \paragraph{Statement.} 
 
 
\end{document}
