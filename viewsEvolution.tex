\documentclass[12pt]{paper}
\usepackage[english]{babel}
\usepackage{indentfirst}
\usepackage{graphicx}
\usepackage{latexsym}
\usepackage{amsmath}
\usepackage{amsthm}
\usepackage{amssymb,amsfonts}
\usepackage{multicol}
\usepackage{xcolor}
\usepackage{changepage}
\usepackage{hyperref}
\usepackage{fancyhdr}
% \usepackage{showkeys}
\usepackage{textcomp}
\usepackage{mathtools}
\usepackage{comment}
\usepackage[normalem]{ulem}
\usepackage{marginnote}
\usepackage[numbers,sort&compress]{natbib}
%%%%%
\mathtoolsset{showonlyrefs=false}

\newcommand{\ga}{\alpha}
\newcommand{\gb}{\beta}
\newcommand{\gam}{\gamma}
\newcommand{\gd}{\delta}
\newcommand{\eps}{\epsilon}
\newcommand{\veps}{\varepsilon}
\newcommand{\gz}{\zeta}
\newcommand{\gt}{\theta}
\newcommand{\gi}{\iota}
\newcommand{\gk}{\kappa}
\newcommand{\gl}{\lambda}
\newcommand{\gs}{\sigma}
\newcommand{\go}{\omega}
\newcommand{\Gam}{\Gamma}
\newcommand{\gD}{\Delta}
\newcommand{\gT}{\Theta}
\newcommand{\gL}{\Lambda}
\newcommand{\gS}{\Sigma}
\newcommand{\gO}{\Omega}

%%%%%%%%%

\newcommand{\pt}[1]{\left( #1\right)}
\newcommand{\pq}[1]{\left[ #1 \right]}
\newcommand{\pg}[1]{\left\{ #1\right\}}
\newcommand{\figref}[1]{\figurename~\ref{#1}}
\newcommand{\red}[1]{\textcolor{red}{#1}}
\newcommand{\blue}[1]{\textcolor{blue}{#1}}
\newcommand{\gray}[1]{\textcolor{gray}{#1}}
% \newcommand{\wikilink}[1]\href{plan.pdf}{Plan}\,(\href{plan.tex}{tex})
% \renewcommand{\baselinestretch}{2} 
%%%%%%%%%
\newcommand{\wikilink}[2] { \href{#1.pdf}{#2}\,(\href{#1.tex}{edit})}

\setlength{\textheight}{20cm}
%\changetext{0cm}{0cm}{0cm}{0cm}{0cm}
\changepage{3.4cm}{5cm}{-2cm}{-2.5cm}{-2.0cm}{-2cm}{0.3cm}{-0.5cm}{0.1cm}
%{length of the text}{width of the text}{}{shift to the right}
%{}{}{}{}{from text to pagenumber}
%\pagenumbering{roman}
\pagestyle{fancy}
\lhead{\bf \today}
\chead{\bf Views on evolution}
\rhead{EG}
\title{Views on evolution -- \today}
\author{EG}
\date{\today}

\begin{document}
 \maketitle
 
 \wikilink{home}{Home}

\wikilink{viewsEvolution}{Self}
 
 \tableofcontents
 \section{Phenotype to Genotype Relation}
\subsection{Phenotype=Genotype and origin of life}

Clearly we can see from the working example\cite{Cliff1993} that adaptive evolution is possible in 
the case when $G=P$. It is not clear, however, if any kind of ``infinite growth of complexity'' is 
possible in such systems. It's very likely, it's not. It is also not clear what ``complexity'' is. 
Nevertheless, we can say that despite parameter space can be very big for $G=P$ it is very simple. 

When it comes to RNA evolution, in principle we can say that if $G=P$ doesn't forbid adaptive 
evolution and von Neumann replicators can function and evolve\cite{Freitas2004}\textbf{(??)} 
without separation of blueprint and machine then \textbf{RNA on its own is capable of adaptive 
evolution}.

In RNA case:
\begin{itemize}
 \item Genotype -- RNA sequence
 \item Phenotype -- RNA fold or, catalytic ability or something else. For the fold it seems, 
thought, we can build function $G\to P$, which is not necessarily reversible: $G\to\{P_i\}$ 
\item Fitness function \textbf{(both bad, both possibly tautological. I would say the tautology 
problem is strong here, the following are more of product of selection than function itself.)}:
  \subitem - Number of RNA copies.
  \subitem - Number of vesicles  with this type of RNA.
\item The system can adapt to:
  \subitem - Skew in monomer distributions
  \subitem - Fluctuation in food supply
  \subitem - New chemicals arriving
\end{itemize}


 \subsection{The system}
 
 An entity: it has a phenotype and genotype and some kind of fitness function. It can be the case, 
when $G=P$ or $G\neq P$
 \paragraph{Example systems:}
\begin{enumerate}
 \item \textbf{Set of vectors.} Vectors are of fixed length and their elements belong to some 
alphabet, for example $\{0,1\}$
  \subitem Genotype -- sequence, Phenotype -- sequence $G=P$ obviously. Fitness function -- 
closeness to some vector by some metric.
  \item \textbf{Protein $G=P$.} A sequence of amino acids folds. Different folds are considered as 
different organisms
  \subitem Genotype -- set of dihedral angles, Phenotype = Genotype. Fitness function -- Free energy
  \item \textbf{Protein $G\neq P$.} A sequence of amino acids folds. Different folds are considered 
as different organisms
  \subitem Genotype -- set of dihedral angles. Phenotype = fitness function = Free energy
  \item \textbf{Robots in a maze.} Robots of certain fixed construction and varying neural networks.
  \subitem Genotype -- Sequence of weights of neural networks. Phenotype -- behavior of the robot. 
Fitness function -- how close the robot will come to a target. It can measure distance from 
itself to the target. $G\neq P$
\end{enumerate}

 
 
 \section{Optimization methods and their relation to biological and artificial evolution}
 
 \subsection{Gradient descent}
 
 \paragraph{Statement.} 

\section{Types of Memory}
\begin{enumerate}
 \item \textbf{Immediate memory.} Let us say we have a population of individuals (molecule for 
example). There are situations for which specie, which already dominates, will continue dominating. 
This is the system which is in a stable attractor or is heading towards the stable attractor. 
\begin{figure}[htb!]
  \centering
  \includegraphics[width=0.35\columnwidth]{pictures/solmem.jpg}(a)
  \includegraphics[width=0.35\columnwidth]{pictures/solmem2.jpg}(b)
  \caption{}
  \label{fig:solmem}
\end{figure}
 If we have a system with simple selection mechanism, then if the individual is in some point of 
fitness landscape, we know that it came from a point with worse fitness and if say there are two 
basins of attraction (see fig \ref{fig:solmem} (a)) then we know that the individual is within a 
certain genotype space. This kind of memory is also realized in a system with competing 
autocatalysts, say A and B. Given some time a better one will dominates the population. However if 
we switch the selection criterion for what is better, or change external condition, in which A is 
performing better than B, to the condition in which B is performing better, then the system will 
switch into another attractor, and when we revert conditions, the system will have no memory what 
soever about the fact that it was in such condition before. 

Thus, this kind of memory is fragile under changes of external conditions with periods bigger 
than it takes the system to reach steady state, if steady state involves extinction of some of the 
individuals in the system. When B is extinct from the system and conditions are hostile to A, then 
the whole system will die, no matter how big or successful it was before.
 
 
 \item \textbf{Linage memory.} A specie has a memory of all its linage members. This kind of memory 
is characteristic of a novelty search algorithm (memory of all the phenotypes) or genetic diversity 
mechanism (memory of all the genotypes). In these algorithms selection is for the new 
phenotype/genotype -- the one, which hasn't been in the linage yet. System avoids being stuck in 
local minima, and is capable of novel functions.

 \item \textbf{Lifetime memory.} Is typical of modern organism, such as, say E.coli. E.coli, if it 
lives, say on glucose, doesn't express proteins, which allow it to live on say lactose. But it has 
a gene corresponding to lactose-proteins -- a memory that it can live on lactose. Thus, despite, 
absence of ``winning'' protein, it can produce it, thus it is resilient to change in condition 
unlike Immediate memory system. Lifetime memory is sort of implementation of linage memory in 
organisms. Through the evolutionary history organisms learned new functions and able to store some 
of the inventions in the DNA.
\end{enumerate}


\section{Volatility as main evolution driver}
\subsection{Conditional of survival}
Ancient cells/polymers/metabolism systems had to survive and better benefit from:
\begin{itemize}
 \item Variation in food supply (as total)
 \item Variation in ratios of different food types
 \item Mutations at every level of informational flow
 \item New arriving chemicals
 \item Variation in temperature
 \item Movement around??
\end{itemize}
Lineage memory implemented as lifetime memory gives resilience, does it give advantage in volatile 
conditions? 
Systems with solution only memory are very fragile \textbf{(are they?)}, because they gradient 
descent quickly \textbf{(do they?)}. 

Extinction events had happened on Earth, rapid changes of conditions had happened; they brought up 
the kind of physical systems which can survive them. If we try to originate life in a steady state, 
we perhaps will reach some results, but it will be nothing close to the life on Earth, we might not 
even notice the result, because we define life based on what we see alive around us. A system in a 
closed bottle can be very successful and capable of adaptive evolution, but if we put it to 
constantly changing environment (often very strongly) will it survive there. What we see as life 
now is a tiny portion of adaptable systems which possibly existed on Earth, and they are the ones, 
which had properties to survive or better benefit from harsh volatility Earth environment was and 
is. 
If the system has lifetime memory of better lineage memory it is more adaptable, and thus more 
resilient. It seems one would want to invent a food-cheap memory storage, where one can amass 
complexity, complexity in the sense of infinitely many stable attractors. A storage which will 
allow to remember old functions and invent new ones.

\paragraph{Copying of I.} A process of copying $I\to I \to I$ through generations can happen with 
and without mistakes:
\begin{eqnarray}
  I_1\to I_1 \to I_1\\
  I_1\to I_1 \to I_2
\end{eqnarray}
Given that mutations are present now and were even more frequent event in the past(no 
error-fixing mechanism), the system would be better of if it \textbf{benefits} from mutations to a 
certain degree. Let's say $I_1$ gives a phenotype $P_1$, which is good for condition $C_1$ and 
$I_2\to P_2@C_2$. Then if the correct mutation $I_1\to I_2$ had happened before change $C_1\to 
C_2$ happened then our organism will survive. If we have many organisms, then some will survive, 
however it is a sort of resilience. If we get to keep both $I_1,\,I_2$ in the system, we would be 
better of -- we loose none at the change. How can we keep both? What will prevent loss of $I$, 
which is not used currently?



\bibliographystyle{apalike}
\bibliography{/data/research/31.mendeleyBibtex/library}
\end{document}
