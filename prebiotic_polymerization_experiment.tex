\documentclass[12pt]{paper}
\usepackage[english]{babel}
\usepackage{indentfirst}
\usepackage{graphicx}
\usepackage{latexsym}
\usepackage{amsmath}
\usepackage{amsthm}
\usepackage{amssymb,amsfonts}
\usepackage{multicol}
\usepackage{xcolor}
\usepackage{changepage}
\usepackage{hyperref}
\usepackage{fancyhdr}
\usepackage{textcomp}
\usepackage{mathtools}
\usepackage{comment}
\usepackage[normalem]{ulem}
\usepackage{marginnote}
\usepackage[numbers,sort&compress]{natbib}
\mathtoolsset{showonlyrefs=false}

\newcommand{\ga}{\alpha}
\newcommand{\gb}{\beta}
\newcommand{\gam}{\gamma}
\newcommand{\gd}{\delta}
\newcommand{\eps}{\epsilon}
\newcommand{\veps}{\varepsilon}
\newcommand{\gz}{\zeta}
\newcommand{\gt}{\theta}
\newcommand{\gi}{\iota}
\newcommand{\gk}{\kappa}
\newcommand{\gl}{\lambda}
\newcommand{\gs}{\sigma}
\newcommand{\go}{\omega}
\newcommand{\Gam}{\Gamma}
\newcommand{\gD}{\Delta}
\newcommand{\gT}{\Theta}
\newcommand{\gL}{\Lambda}
\newcommand{\gS}{\Sigma}
\newcommand{\gO}{\Omega}

%%%%%%%%%

\newcommand{\pt}[1]{\left( #1\right)}
\newcommand{\pq}[1]{\left[ #1 \right]}
\newcommand{\pg}[1]{\left\{ #1\right\}}
\newcommand{\figref}[1]{\figurename~\ref{#1}}
\newcommand{\red}[1]{\textcolor{red}{#1}}
\newcommand{\blue}[1]{\textcolor{blue}{#1}}
\newcommand{\gray}[1]{\textcolor{gray}{#1}}
\newcommand{\wikilink}[2] { \href{#1.pdf}{#2}\,(\href{#1.tex}{edit})}

\setlength{\textheight}{20cm}
\changepage{2.5cm}{3.0cm}{-4cm}{-1.0cm}{-2cm}{-2cm}{-0.6cm}{0.5cm}{0cm}
\pagestyle{fancy}
\lhead{\bf \today}
\chead{\bf Prebiotic polymerization: experiment}
\rhead{EG}
\title{Prebiotic polymerization: experiment -- \today}
\author{EG}
\date{\today}

\begin{document}
 \maketitle
\wikilink{home}{Home}
 
\wikilink{literature\_experiment}{Literature: Experiment}


\begin{enumerate}
 \item  There has been some success at synthesis of macromolecules
 in the laboratory under aqueous conditions using exotic condensing
reagents \cite{Joyce1989}??. But such chemistry is
far from the facile reactions that likely sparked life. 

\item  ``A major stumbling block is the fact that biological polymers generally are formed by 
dehydration, the removal of water, to
form a peptide bond in the case of protein or a phosphodiester bond in the case of nucleic acids. 
However, because of the high concentration of water (55 M) in a fully
aqueous environment, hydrolysis, not polymerization,
 is overwhelmingly favored. The origin of life seems inconsistent with fully aqueous chemistry, 
particularly at high temperature.'' \cite{Pace1991}

\item Problems with RNA world: ``If it ever existed, however, the RNA world was not ex-
posed to free solution. The ribose 21-OH group that renders
RNA, in contrast to DNA, catalytic also renders the RNA
chain particularly susceptible to hydrolysis. The RNA
phosphodiester
 in aqueous solution is highly labile to hy-
drolysis promoted by the ribose 2'-OH group, which forms
the 2',3'-cyclic phosphate and breaks the RNA chain. This
reaction is accelerated by high pH, high temperature, and
the presence of divalent (or other multivalent) cations,
which bind to and polarize phosphates, enhancing reactiv-
ity. The early hydrosphere would have contained substan-
tial concentrations of many multivalent cations. RNA mole-
cules of much complexity could not have survived if
exposed to solution in that environment, particularly if at
high temperatures.''\cite{Pace1991}


\end{enumerate}

 
 
 
 
 
 
 
 
 

   \bibliography{/data/research/31.mendeleyBibtex/library}
  \bibliographystyle{unsrt}  
 
\end{document}
