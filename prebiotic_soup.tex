\documentclass[12pt]{paper}
\usepackage[english]{babel}
\usepackage{indentfirst}
\usepackage{graphicx}
\usepackage{latexsym}
\usepackage{amsmath}
\usepackage{amsthm}
\usepackage{amssymb,amsfonts}
\usepackage{multicol}
\usepackage{xcolor}
\usepackage{changepage}
\usepackage{hyperref}
\usepackage{fancyhdr}
\usepackage{textcomp}
\usepackage{mathtools}
\usepackage{comment}
\usepackage[normalem]{ulem}
\usepackage{marginnote}
\usepackage[numbers,sort&compress]{natbib}
\mathtoolsset{showonlyrefs=false}

\newcommand{\ga}{\alpha}
\newcommand{\gb}{\beta}
\newcommand{\gam}{\gamma}
\newcommand{\gd}{\delta}
\newcommand{\eps}{\epsilon}
\newcommand{\veps}{\varepsilon}
\newcommand{\gz}{\zeta}
\newcommand{\gt}{\theta}
\newcommand{\gi}{\iota}
\newcommand{\gk}{\kappa}
\newcommand{\gl}{\lambda}
\newcommand{\gs}{\sigma}
\newcommand{\go}{\omega}
\newcommand{\Gam}{\Gamma}
\newcommand{\gD}{\Delta}
\newcommand{\gT}{\Theta}
\newcommand{\gL}{\Lambda}
\newcommand{\gS}{\Sigma}
\newcommand{\gO}{\Omega}

%%%%%%%%%

\newcommand{\pt}[1]{\left( #1\right)}
\newcommand{\pq}[1]{\left[ #1 \right]}
\newcommand{\pg}[1]{\left\{ #1\right\}}
\newcommand{\figref}[1]{\figurename~\ref{#1}}
\newcommand{\red}[1]{\textcolor{red}{#1}}
\newcommand{\blue}[1]{\textcolor{blue}{#1}}
\newcommand{\gray}[1]{\textcolor{gray}{#1}}
\newcommand{\wikilink}[2] { \href{#1.pdf}{#2}\,(\href{#1.tex}{edit})}

\setlength{\textheight}{20cm}
\changepage{2.5cm}{3.0cm}{-4cm}{-1.0cm}{-2cm}{-2cm}{-0.6cm}{0.5cm}{0cm}
\pagestyle{fancy}
\lhead{\bf \today}
\chead{\bf Prebiotic soup theory}
\rhead{EG}
\title{Prebiotic soup theory -- \today}
\author{EG}
\date{\today}

\begin{document}
 \maketitle
\wikilink{home}{Home}

\wikilink{ideas}{Ideas}

\section{The idea of prebiotic soup}
``The concept that life arose from a prebiotic soup or primeval broth that covered the Earth is 
generally attributed to Oparin \cite{Oparin1952} and Haldane \cite{Haldane1929}. The theory 
received support from Miller’s \cite{Miller1953} demonstration that organic molecules could be
obtained by the action of simulated lightning on a mixture of the gases $CH_4$, $NH_3$ and $H_2$, 
which were thought at that time to represent Earth’s earliest atmosphere. The organic compounds 
that were measured included hydrogen cyanide (HCN), aldehydes, amino acids, oil and tar. Additional 
amino acids were produced by Strecker synthesis\footnote{The Strecker amino acid synthesis, devised 
by Adolph Strecker, is a series of chemical reactions that synthesize an amino acid from an 
aldehyde or ketone. The aldehyde is condensed with ammonium chloride in the presence of 
potassium cyanide to form an $\ga$-aminonitrile, which is subsequently hydrolyzed to give the 
desired amino acid.}
 through the hydrolysis of the reaction products of 
HCN, ammonium chloride and aldehydes, and in later experiments polymerization of HCN
produced the nucleic acid bases adenine and guanine. ''\cite{Martin2008}

However there are difficulties with the polymerization of the precursor molecules. First of all, 
RNA molecule favours hydrolysis over polymerization in aqueous solutions and it wouldn't survive 
if exposed to solution, especially at high temperatures\cite{Pace1991}. The similar situation 
holds for proteins \cite{Lambert2008}

Several mechanisms to go around the problem were proposed: ``evaporation of tidal pools, 
adsorption to clays, concentration in ice through eutectic melts and giant oil
slicks. Temperature cycling might also have been a factor in peptide production, although cold to 
freezing conditions are now considered to be more favourable for prebiotic 
soup\cite{Bada2004}.''\cite{Martin2008}




\bibliography{/data/research/31.mendeleyBibtex/library}
\bibliographystyle{unsrt}  
\end{document}
