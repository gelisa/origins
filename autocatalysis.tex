\documentclass[12pt]{paper}
\usepackage[english]{babel}
\usepackage{indentfirst}
\usepackage{graphicx}
\usepackage{latexsym}
\usepackage{amsmath}
\usepackage{amsthm}
\usepackage{amssymb,amsfonts}
\usepackage{multicol}
\usepackage{xcolor}
\usepackage{changepage}
\usepackage{hyperref}
\usepackage{fancyhdr}
\usepackage{textcomp}
\usepackage{mathtools}
\usepackage{comment}
\usepackage[normalem]{ulem}
\usepackage{marginnote}
\usepackage[numbers,sort&compress]{natbib}
\mathtoolsset{showonlyrefs=false}

\newcommand{\ga}{\alpha}
\newcommand{\gb}{\beta}
\newcommand{\gam}{\gamma}
\newcommand{\gd}{\delta}
\newcommand{\eps}{\epsilon}
\newcommand{\veps}{\varepsilon}
\newcommand{\gz}{\zeta}
\newcommand{\gt}{\theta}
\newcommand{\gi}{\iota}
\newcommand{\gk}{\kappa}
\newcommand{\gl}{\lambda}
\newcommand{\gs}{\sigma}
\newcommand{\go}{\omega}
\newcommand{\Gam}{\Gamma}
\newcommand{\gD}{\Delta}
\newcommand{\gT}{\Theta}
\newcommand{\gL}{\Lambda}
\newcommand{\gS}{\Sigma}
\newcommand{\gO}{\Omega}

%%%%%%%%%

\newcommand{\pt}[1]{\left( #1\right)}
\newcommand{\pq}[1]{\left[ #1 \right]}
\newcommand{\pg}[1]{\left\{ #1\right\}}
\newcommand{\figref}[1]{\figurename~\ref{#1}}
\newcommand{\red}[1]{\textcolor{red}{#1}}
\newcommand{\blue}[1]{\textcolor{blue}{#1}}
\newcommand{\gray}[1]{\textcolor{gray}{#1}}
\newcommand{\wikilink}[2] { \href{#1.pdf}{#2}\,(\href{#1.tex}{edit})}

\setlength{\textheight}{20cm}
\changepage{2.5cm}{3.0cm}{-4cm}{-1.0cm}{-2cm}{-2cm}{-0.6cm}{0.5cm}{0cm}
\pagestyle{fancy}
\lhead{\bf \today}
\chead{\bf Autocatalysis}
\rhead{EG}
\title{Autocatalysis -- \today}
\author{EG}
\date{\today}

\begin{document}
 \maketitle
\wikilink{home}{Home}

\wikilink{ideas}{Ideas}
 
\section{Papers}
\subsection{Autocatalytic Sets and the Origin of Life by
Wim Hordijk, Jotun Hein and Mike Steel\cite{Hordijk2010}}

\begin{itemize}
 \item  Origin of life research: common element seems to be the emergence of an autocatalytic set 
or cycle at some stage. 
\item Hypercycle is a subset of autocatalytic sets

\item Kauffman's idea that in  sufficiently complex chemical reaction systems, autocatalytic
sets will arise almost inevitably was disputed by pointing out that this requires an
(unrealistic) exponential growth in catalytic activity with increasing system size. It was argued 
back by Kauffman?

\item It is at least plausible that autocatalytic sets indeed
played a role in the emergence of proteins and DNA from an RNA world
\end{itemize}

\section{Against autocatalysis}
29.
 Lifson, S. On the crucial stages in the origin of animate matter. J. Mol. Evol. 1997, 44, 1–8.

\section{Experimental support}

\paragraph{Simple autocatalytic sets}
31.
 Sievers, D.; von Kiedrowski, G. Self-replication of complementary nucleotide-based oligomers.
Nature 1994, 369, 221–224.\\
32.
 Ashkenasy, G.; Jegasia, R.; Yadav, M.; Ghadiri, M.R. Design of a directed molecular network.
PNAS 2004, 101, 10872–10877.\\
33.
 Hayden, E.J.; von Kieddrowski, G.; Lehman, N. Systems chemistry on ribozyme self-construction:
Evidence for anabolic autocatalysis in a recombination network. Angew. Chem. Int. Ed. 2008,
120, 8552–8556.










\bibliography{/data/research/31.mendeleyBibtex/library}
\bibliographystyle{unsrt}   
\end{document}
