\documentclass[12pt]{paper}
\usepackage[english]{babel}
\usepackage{indentfirst}
\usepackage{graphicx}
\usepackage{latexsym}
\usepackage{amsmath}
\usepackage{amsthm}
\usepackage{amssymb,amsfonts}
\usepackage{multicol}
\usepackage{xcolor}
\usepackage{changepage}
\usepackage{hyperref}
\usepackage{fancyhdr}
\usepackage{textcomp}
\usepackage{mathtools}
\usepackage{comment}
\usepackage[normalem]{ulem}
\usepackage{marginnote}
\usepackage[numbers,sort&compress]{natbib}
\mathtoolsset{showonlyrefs=false}

\newcommand{\ga}{\alpha}
\newcommand{\gb}{\beta}
\newcommand{\gam}{\gamma}
\newcommand{\gd}{\delta}
\newcommand{\eps}{\epsilon}
\newcommand{\veps}{\varepsilon}
\newcommand{\gz}{\zeta}
\newcommand{\gt}{\theta}
\newcommand{\gi}{\iota}
\newcommand{\gk}{\kappa}
\newcommand{\gl}{\lambda}
\newcommand{\gs}{\sigma}
\newcommand{\go}{\omega}
\newcommand{\Gam}{\Gamma}
\newcommand{\gD}{\Delta}
\newcommand{\gT}{\Theta}
\newcommand{\gL}{\Lambda}
\newcommand{\gS}{\Sigma}
\newcommand{\gO}{\Omega}

%%%%%%%%%

\newcommand{\pt}[1]{\left( #1\right)}
\newcommand{\pq}[1]{\left[ #1 \right]}
\newcommand{\pg}[1]{\left\{ #1\right\}}
\newcommand{\figref}[1]{\figurename~\ref{#1}}
\newcommand{\red}[1]{\textcolor{red}{#1}}
\newcommand{\blue}[1]{\textcolor{blue}{#1}}
\newcommand{\gray}[1]{\textcolor{gray}{#1}}
\newcommand{\wikilink}[2] { \href{#1.pdf}{#2}\,(\href{#1.tex}{edit})}

\setlength{\textheight}{20cm}
\changepage{2.5cm}{3.0cm}{-4cm}{-1.0cm}{-2cm}{-2cm}{-0.6cm}{0.5cm}{0cm}
\pagestyle{fancy}
\lhead{\bf \today}
\chead{Nowak-Chen \bf }
\rhead{EG}
\title{Nowak-Chen Model -- \today}
\author{EG}
\date{\today}

\begin{document}
 \maketitle
\wikilink{home}{Home}

\wikilink{research\_calculations}{Research: Calculations}

\wikilink{nowak-chen}{Self}

\tableofcontents


\section{Parameters of the model}
Model was presented in the papers \cite{nowak2008prevolutionary,Ohtsuki2009,Chen2012}
\begin{itemize}
 \item We assume that there are enough of activated monomers in the system, so that their 
concentrations are constant. This way we don't have to track them in the simulations.

\item Polymers can therefore spontaneously grow with the rate $\ga$. Without loss of generality we 
can put this parameter equal 1; all other rates with be relative to the growth rate in this case.

\item Activated monomers decay into regular ones with the rate $a\gg1$. This 
imitates input of food into the system. It is safe to assume that decay of activated monomers 
happens much faster than spontaneous growth of polymers. Therefore we explore values of $a\propto 
10^2,10^3\ga$
\item Dilution parameter $d$ mimics cell division and loss of the matter because of that. From 
\ref{sec:nowak-steady} we see that total mass of the system is  $ M\propto\frac{a}{\ga}, \qquad 
d\approx \ga\quad \mbox{or}\quad d\gg \ga$ and $M\propto \frac{a}{\ga}\frac{d}{2\ga} ,\qquad 
d\ll\ga$. Therefore we explore valued of $d$ from $\propto 0.01\ga$ to $\propto 1\ga$. Given 
values of $a$ we'll explore various populations from $\propto 10^?$ to $\propto 10^5$ monomers per 
cell.

\item Hydrolysis has constant rate $d_h$ per bond; it varies from $\propto 0.01$ to $\propto 1$. 
\subitem For instance, the
uncatalyzed hydrolysis of glycylglycine under neutral conditions
at 25 1C proceeds with a rate constant of $6.3  10^{-11} M^{-1} s^{-1}$
(i.e. an half life of 350 y), whereas values of $9.3 10^{-11} s^{-1}$
have been reported for glycylvaline at 37 C under neutral
conditions\cite{Smith1998}, and $1.3  10^{-10} s^{-1} $ for benzoylglycylphenylalanine
(t1/2 = 128 y)\cite{Bryant1996}.
\subitem kinetics are depending on the position of the residues within the peptide chain 
\cite{Danger2012}

\item Folding and unfolding reactions happen very quickly with the rates $k_{unf}\gg1$ and 
$k_{unf}\cdot\exp(E_{native}/kT)$ correspondingly.

\item Catalysis rate is proportional to the exponent of hydrophobic energy $E_h$ and number of 
contacting hydrophobes $n_c$: $\ga\cdot\exp(E_{h}\cdot n_{c}/kT)$

\item Some experiments also include aggregation reactions for the long hydrophobic chains.
\end{itemize}

\section{Kinetics}
We enumerate all the polymers, so that $x_i$ is population of $i^{th}$ monomer, and $x_{i'}$ is a 
population of its precursor.

Equations are:
  \begin{eqnarray}
   \mbox{One mers:}&& \dot{x_i}=a-2\ga x_i-dx_i \\
     \mbox{2+ mers:}&& \dot{x_i}=\ga x_{i'}-(2\ga+d)x_i
  \end{eqnarray}

\subsection{Steady State Kinetics}\label{sec:nowak-steady}
Steady state: $\dot{x_i}=0$
  \begin{eqnarray}
   \mbox{One mers:}&& 0=a-2\ga x_i-dx_i \\
     \mbox{2+ mers:}&& 0=\ga x_{i'}-(2\ga+d)x_i
  \end{eqnarray}
  So we have:
   \begin{eqnarray}
   \mbox{One mers:}&& x_i=\frac{a}{2\ga+d} \\
     \mbox{2+ mers:}&& x_i=\frac{\ga}{2\ga+d}x_{i'}
  \end{eqnarray}   
 Therefore for every sequence of length $l$ we get:
   \begin{equation}
   \boxed{ x_l=\frac{a}{\ga}\pt{\frac{\ga}{2\ga+d}}^l}
   \end{equation} 

Population of all the sequence of length $l$ is therefore:
  \begin{equation}
    p_l=\frac{a}{\ga}\pt{\frac{\ga}{2\ga+d}}^l2^l=\frac{a}{\ga}\pt{\frac{2\ga}{2\ga+d}}^l=
    \frac{a}{\ga}\pt{\frac{1}{1+d/2\ga}}^l
  \end{equation} 
If we denote $x\equiv\frac{d}{2\ga}$, population of all the sequences of length $l$ will be:
\begin{equation}
\boxed{ p_l=\frac{a}{\ga}\pt{\frac{1}{1+x}}^l}
\end{equation} 
Total mass of all the sequences is:
\begin{equation}
 M=\sum_{l=0}^{\infty}lp_l
\end{equation} 
\begin{equation}
 M=\sum_{l=0}^{\infty}\frac{a}{\ga}l\pt{\frac{1}{1+x}}^l
\end{equation} 
According to \cite{Gradstein1980} the sum will be
\begin{equation}
 M=\frac{a}{\ga}\frac{\frac{1}{1+x}}{\pt{1-\frac{1}{1+x}}^2}=\frac{a}{\ga}\pt{\frac{1+x}{x}}
\end{equation}
Therefore total mass is:
  \begin{equation}
   M=\frac{a}{\ga}\pt{1+\frac{1}{x}}
  \end{equation} 
Remember that $x=d/2\ga$. It means that values of $d$ $d\approx \ga$ or $d\gg \ga$ produce total 
masses 
\begin{equation}
 M\propto\frac{a}{\ga}, \qquad d\approx \ga\quad \mbox{or}\quad d\gg \ga
\end{equation} 
while very small values of $d:\,d\ll\ga$ produce total masses 
\begin{equation}
M\propto \frac{a}{\ga}\frac{d}{2\ga} ,\qquad d\ll\ga
\end{equation}


 \bibliography{/data/research/31.mendeleyBibtex/library}
  \bibliographystyle{unsrt}


















\end{document}