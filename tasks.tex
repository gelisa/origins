\documentclass[12pt]{paper}
\usepackage[english]{babel}
\usepackage{indentfirst}
\usepackage{graphicx}
\usepackage{latexsym}
\usepackage{amsmath}
\usepackage{amsthm}
\usepackage{amssymb,amsfonts}
\usepackage{multicol}
\usepackage{xcolor}
\usepackage{changepage}
\usepackage{hyperref}
\usepackage{fancyhdr}
\usepackage{textcomp}
\usepackage{mathtools}
\usepackage{comment}
\usepackage[normalem]{ulem}
\usepackage{marginnote}
\usepackage[numbers,sort&compress]{natbib}
\mathtoolsset{showonlyrefs=false}

\newcommand{\ga}{\alpha}
\newcommand{\gb}{\beta}
\newcommand{\gam}{\gamma}
\newcommand{\gd}{\delta}
\newcommand{\eps}{\epsilon}
\newcommand{\veps}{\varepsilon}
\newcommand{\gz}{\zeta}
\newcommand{\gt}{\theta}
\newcommand{\gi}{\iota}
\newcommand{\gk}{\kappa}
\newcommand{\gl}{\lambda}
\newcommand{\gs}{\sigma}
\newcommand{\go}{\omega}
\newcommand{\Gam}{\Gamma}
\newcommand{\gD}{\Delta}
\newcommand{\gT}{\Theta}
\newcommand{\gL}{\Lambda}
\newcommand{\gS}{\Sigma}
\newcommand{\gO}{\Omega}

%%%%%%%%%

\newcommand{\pt}[1]{\left( #1\right)}
\newcommand{\pq}[1]{\left[ #1 \right]}
\newcommand{\pg}[1]{\left\{ #1\right\}}
\newcommand{\figref}[1]{\figurename~\ref{#1}}
\newcommand{\red}[1]{\textcolor{red}{#1}}
\newcommand{\blue}[1]{\textcolor{blue}{#1}}
\newcommand{\gray}[1]{\textcolor{gray}{#1}}
\newcommand{\wikilink}[2] { \href{#1.pdf}{#2}\,(\href{#1.tex}{edit})}

\setlength{\textheight}{20cm}
\changepage{3.4cm}{5cm}{-2cm}{-2.5cm}{-2.0cm}{-2cm}{0.3cm}{-0.5cm}{0.1cm}
\pagestyle{fancy}
\lhead{\bf \today}
\chead{\bf }
\rhead{EG}
\title{Tasks  -- \today}
\author{EG}
\date{\today}

\begin{document}
 \maketitle
\wikilink{home}{Home}

\section{HP world article}
\begin{enumerate}
 \item See length distribution of simulations 7-13: in resources/expVSpol folder there are some 
peculiar pictures
\end{enumerate}

\section{Two polymers}
\begin{enumerate}
 \item Look futher into McMullin's work to look if he implemented his Genetic Relativism idea
 \item FInd for roots of Taleb's statement about mathematics of evolution.
 \item Evolvability wiki article
 \item Lipson2005 also might be useful
\end{enumerate}

\section{In general, maybe}
\begin{enumerate}
\item Complexity:
\subitem 503. Christof Adami, ``What is complexity?''
BioEssays 24 (2002) : 1085-1094; \\
http://www.krl.caltech.edu/~adami/BE2002.pdf.
\subitem 504. Christof Adami, Charles Ofria, Travis C. Collier, “Evolution of biological
complexity,” Proc. Nat. Acad. Sci. (USA) 97 (26 May 2000): 4463-4468; \\
http://xxx.lanl.gov/PS\_cache/physics/pdf/0005/0005074.pdf.
\end{enumerate}

\end{document}