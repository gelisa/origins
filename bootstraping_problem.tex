\documentclass[12pt]{paper}
\usepackage[english]{babel}
\usepackage{indentfirst}
\usepackage{graphicx}
\usepackage{latexsym}
\usepackage{amsmath}
\usepackage{amsthm}
\usepackage{amssymb,amsfonts}
\usepackage{multicol}
\usepackage{xcolor}
\usepackage{changepage}
\usepackage{hyperref}
\usepackage{fancyhdr}
% \usepackage{showkeys}
\usepackage{textcomp}
\usepackage{mathtools}
\usepackage{comment}
\usepackage[normalem]{ulem}
\usepackage{marginnote}
\usepackage[numbers,sort&compress]{natbib}
%%%%%
\mathtoolsset{showonlyrefs=false}

\newcommand{\ga}{\alpha}
\newcommand{\gb}{\beta}
\newcommand{\gam}{\gamma}
\newcommand{\gd}{\delta}
\newcommand{\eps}{\epsilon}
\newcommand{\veps}{\varepsilon}
\newcommand{\gz}{\zeta}
\newcommand{\gt}{\theta}
\newcommand{\gi}{\iota}
\newcommand{\gk}{\kappa}
\newcommand{\gl}{\lambda}
\newcommand{\gs}{\sigma}
\newcommand{\go}{\omega}
\newcommand{\Gam}{\Gamma}
\newcommand{\gD}{\Delta}
\newcommand{\gT}{\Theta}
\newcommand{\gL}{\Lambda}
\newcommand{\gS}{\Sigma}
\newcommand{\gO}{\Omega}

%%%%%%%%%

\newcommand{\pt}[1]{\left( #1\right)}
\newcommand{\pq}[1]{\left[ #1 \right]}
\newcommand{\pg}[1]{\left\{ #1\right\}}
\newcommand{\figref}[1]{\figurename~\ref{#1}}
\newcommand{\red}[1]{\textcolor{red}{#1}}
\newcommand{\blue}[1]{\textcolor{blue}{#1}}
\newcommand{\gray}[1]{\textcolor{gray}{#1}}
% \newcommand{\wikilink}[1]\href{plan.pdf}{Plan}\,(\href{plan.tex}{tex})
% \renewcommand{\baselinestretch}{2} 
%%%%%%%%%
\newcommand{\wikilink}[2] { \href{#1.pdf}{#2}\,(\href{#1.tex}{edit})}

\setlength{\textheight}{20cm}
%\changetext{0cm}{0cm}{0cm}{0cm}{0cm}
\changepage{3.4cm}{5cm}{-2cm}{-2.5cm}{-2.0cm}{-2cm}{0.3cm}{-0.5cm}{0.1cm}
%{length of the text}{width of the text}{}{shift to the right}
%{}{}{}{}{from text to pagenumber}
%\pagenumbering{roman}
\pagestyle{fancy}
\lhead{\bf \today}
\chead{\bf Bootstraping}
\rhead{EG}
\title{Bootstraping problem -- \today}
\author{EG}
\date{\today}

\begin{document}
 \maketitle
 
 \wikilink{home}{Home}

\wikilink{bootstraping_problem}{Self}
 
 \tableofcontents
 
\paragraph{The problem.} The evolutionary boot-strapping problem: in the von Neumann framework, at 
least, $u_0$ is already a very complicated entity. It certainly seems implausible that it could 
occur
spontaneously or by chance. Similarly, in real biology, the modern (self-consistent!)
genetic system could not have plausibly arisen by chance. It seems that we must
therefore assume that something like $u_0$ (or a full blown genetic system) must itself
be the product of an extended evolutionary process. Of course, the problem with
this—and a major part of von Neumann’s own result—is that it seems that something
like a genetic system is a pre-requisite to any such evolutionary 
process.\footnote{\label{note1}Full citation from \cite{McMullin2000}}


\paragraph{Genetic Relativism as possible solution}
\textit{Genetic Relativism} \cite{McMullin1992}(Section 5.4) envisages that the mapping 
between genotype (description tape) and phenotype (self-reproducing automaton) is not fixed or 
absolute but may vary from one organism (automaton) to another.
 \textit{Genetic Absolutism} says that genotype defines phenotype in one way.


\paragraph{Before first replicator/general constructor}
Prima facie, our solution based on Genetic Absolutism may seem to imply that a general constructive 
automaton (i.e., capable of constructing a very wide range of target machines) is a pre-requisite to 
any evolutionary growth of complexity. It is not. Indeed, we may say that, if such an implication 
were present, we should probably have to regard our solution as defective, for it would entirely beg 
the question of how such a relatively complex entity as $u_0$ (or something fairly close to it) 
could arise in the first place. Conversely, once we recognise the possibility of evolution within 
the framework of Genetic Relativism, we can at least see how such prior elaboration of the powers of 
the constructive automata could occur ``in principle''; this insight remains valid, at least as a
coherent conjecture, even if we have not demonstrated it in operation. This has a possible advantage 
in relation to the solution of von Neumann’s problem in that it may permit us to
work, initially at least, with significantly more primitive constructive automata as the bases
of our self-reproducers.


 
 
\paragraph{Enhancing Evolvobility}
 Genetic Absolutism views all the self-reproducers under investigation as con-
nected by a single ``genetic network'' of mutational changes. This is sufficient to solve von
Neumann’s problem, as stated, which called only for exhibiting the possibility of mutational
growth of complexity. In practice, however, we are interested in this as a basis for a Darwinian 
growth of complexity. Roughly speaking, this can only occur, if at all, along paths
in the genetic network which lead ``uphill'' in terms of ``fitness''. If the genetic network is
fixed then this may impose severe limits on the practical paths of Darwinian evolution (and
thus on the practical growth of complexity). Again, once we recognise the possibility of
evolution within a framework of Genetic Relativism—which offers the possibility, in effect,
of changing, or jumping between, different genetic networks—the practical possibilities for
the (Darwinian) growth of complexity are evidently greatly increased.\footnotemark[1]


 

  \bibliography{library}
  \bibliographystyle{unsrt}
 
\end{document}
