\documentclass[12pt]{paper}
\usepackage[english]{babel}
\usepackage{indentfirst}
\usepackage{graphicx}
\usepackage{latexsym}
\usepackage{amsmath}
\usepackage{amsthm}
\usepackage{amssymb,amsfonts}
\usepackage{multicol}
\usepackage{xcolor}
\usepackage{changepage}
\usepackage{hyperref}
\usepackage{fancyhdr}
\usepackage{textcomp}
\usepackage{mathtools}
\usepackage{comment}
\usepackage[normalem]{ulem}
\usepackage{marginnote}
\usepackage[numbers,sort&compress]{natbib}
\mathtoolsset{showonlyrefs=false}

\newcommand{\ga}{\alpha}
\newcommand{\gb}{\beta}
\newcommand{\gam}{\gamma}
\newcommand{\gd}{\delta}
\newcommand{\eps}{\epsilon}
\newcommand{\veps}{\varepsilon}
\newcommand{\gz}{\zeta}
\newcommand{\gt}{\theta}
\newcommand{\gi}{\iota}
\newcommand{\gk}{\kappa}
\newcommand{\gl}{\lambda}
\newcommand{\gs}{\sigma}
\newcommand{\go}{\omega}
\newcommand{\Gam}{\Gamma}
\newcommand{\gD}{\Delta}
\newcommand{\gT}{\Theta}
\newcommand{\gL}{\Lambda}
\newcommand{\gS}{\Sigma}
\newcommand{\gO}{\Omega}

%%%%%%%%%

\newcommand{\pt}[1]{\left( #1\right)}
\newcommand{\pq}[1]{\left[ #1 \right]}
\newcommand{\pg}[1]{\left\{ #1\right\}}
\newcommand{\figref}[1]{\figurename~\ref{#1}}
\newcommand{\red}[1]{\textcolor{red}{#1}}
\newcommand{\blue}[1]{\textcolor{blue}{#1}}
\newcommand{\gray}[1]{\textcolor{gray}{#1}}
\newcommand{\wikilink}[2] { \href{#1.pdf}{#2}\,(\href{#1.tex}{edit})}

\setlength{\textheight}{20cm}
\changepage{2.5cm}{3.0cm}{-4cm}{-1.0cm}{-2cm}{-2cm}{-0.6cm}{0.5cm}{0cm}
\pagestyle{fancy}
\lhead{\bf \today}
\chead{\bf RNA World}
\rhead{EG}
\title{RNA World\today}
\author{EG}
\date{\today}

\begin{document}
 \maketitle
\wikilink{home}{Home}

\wikilink{ideas}{Ideas}

\section{Origination of the RNA world idea}

For a while people thought that RNA doesn't have enzymatic activity. And proteins cannot store 
genetic information. Assumption was that in order to store information a carrier must be one 
dimensional for ease of copying. Proteins are however have complex 3D structure and don't follow 
demands of copying mechanisms. 
 Because of that people thought life had to originate with both polymers 
simultaneously. One responsible for information and another for function. However at some point 
people discovered that RNA can have enzymatic activity:\
\begin{enumerate}
 \item RNA molecule in E. coli: ribonuclease-P cuts phosphodiester bonds during the maturation of 
the transfer RNA molecule\cite{Guerrier-Takada1983,Guerrier-Takada1984}
\item In Tetrahymena ribosomal RNA contains a self-splicing exon \cite{Cech1981,Kruger1982}
\end{enumerate}
So people thought that if there are two enzymatic activities associated with RNA there might be 
more. Later more self-splicinge introns were found \cite{Cech1986}.

People contemplated an \textit{RNA world}: a world which has only RNA molecules which synthesize 
and catalyze themselves. An assumption was made that a self-splicing intron, which can cut itself 
out of RNA can have a reverse reaction and put itself back into RNA in the proper place. 
Self-inserting introns can this way have a major evolutionary advantage -- recombination: the 
ability to produce new combination of genes. RNA world also solves dichotomy between DNA and 
proteins. 

Stages of evolution would be the following according to the original idea:
\begin{enumerate}
 \item ??
 \item RNA molecules perform catalytic activities to  assemble itself from prebiotic soup.
 \item RNA molecules evolve in self-replicating patterns, using recombination and mutation.
 \item By using RNA cofactors the develop the whole range of enzymatic activity.
 \item RNA molecules begin to synthesize proteins: first by developing RNA adapter molecules that 
can bind activated amino acids and then by arranging them according to RNA template using RNA 
molecules such as ribosome core.
\item The first proteins would be better enzymes (\textit{right away}) and they perform the same 
reactions as RNA not of different nature. Therefore they will eventually dominate.
\end{enumerate}


This section is based on \cite{Gilbert1986}

\section{Rise of the idea. Experiments}

\section{Fall of the idea. Scientist are getting disappointed and shift their attention away from 
RNA}


  \bibliography{/data/research/31.mendeleyBibtex/library}
  \bibliographystyle{unsrt} 
\end{document}
