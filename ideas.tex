\documentclass[12pt]{paper}
\usepackage[english]{babel}
\usepackage{indentfirst}
\usepackage{graphicx}
\usepackage{latexsym}
\usepackage{amsmath}
\usepackage{amsthm}
\usepackage{amssymb,amsfonts}
\usepackage{multicol}
\usepackage{xcolor}
\usepackage{changepage}
\usepackage{hyperref}
\usepackage{fancyhdr}
\usepackage{textcomp}
\usepackage{mathtools}
\usepackage{comment}
\usepackage[normalem]{ulem}
\usepackage{marginnote}
\usepackage[numbers,sort&compress]{natbib}
\mathtoolsset{showonlyrefs=false}

\newcommand{\ga}{\alpha}
\newcommand{\gb}{\beta}
\newcommand{\gam}{\gamma}
\newcommand{\gd}{\delta}
\newcommand{\eps}{\epsilon}
\newcommand{\veps}{\varepsilon}
\newcommand{\gz}{\zeta}
\newcommand{\gt}{\theta}
\newcommand{\gi}{\iota}
\newcommand{\gk}{\kappa}
\newcommand{\gl}{\lambda}
\newcommand{\gs}{\sigma}
\newcommand{\go}{\omega}
\newcommand{\Gam}{\Gamma}
\newcommand{\gD}{\Delta}
\newcommand{\gT}{\Theta}
\newcommand{\gL}{\Lambda}
\newcommand{\gS}{\Sigma}
\newcommand{\gO}{\Omega}

%%%%%%%%%

\newcommand{\pt}[1]{\left( #1\right)}
\newcommand{\pq}[1]{\left[ #1 \right]}
\newcommand{\pg}[1]{\left\{ #1\right\}}
\newcommand{\figref}[1]{\figurename~\ref{#1}}
\newcommand{\red}[1]{\textcolor{red}{#1}}
\newcommand{\blue}[1]{\textcolor{blue}{#1}}
\newcommand{\gray}[1]{\textcolor{gray}{#1}}
\newcommand{\wikilink}[2] { \href{#1.pdf}{#2}\,(\href{#1.tex}{edit})}

\setlength{\textheight}{20cm}
\changepage{2.5cm}{3.0cm}{-4cm}{-1.0cm}{-2cm}{-2cm}{-0.6cm}{0.5cm}{0cm}
\pagestyle{fancy}
\lhead{\bf \today}
\chead{\bf Ideas}
\rhead{EG}
\title{Ideas\today}
\author{EG}
\date{\today}

\begin{document}
 \maketitle
\wikilink{home}{Home}

\wikilink{ideas}{Self}
 
\tableofcontents


\section{General}
\begin{itemize}
 \item \wikilink{compartment}{Compartmentalization} \red{no file} Is compartmentalization 
necessary or not?
\item \wikilink{prebiotic\_soup}{Prebiotic Soup}: life originated from prebiotic soup.
\item \wikilink{autocatalysis}{Autocatalysis}: Autocatalysis and autocatalytic sets played 
important role in the origin of life
\item \wikilink{metabolism}{Metabolism first approach}
\end{itemize}


\section{Prebiotic polymerization}
\subsection{From literature}
\begin{itemize}
\item \wikilink{rna\_world}{RNA-world idea}
 \item  \wikilink{peptide\_world}{Peptides could form prebiotically} \red{no file}
 \item \wikilink{prebiotic\_aa}{Amino acids could be formed prebiotically} \red{no file}
 \item \wikilink{prebiotic\_nucleotides}{Nucleotides could be synthesized prebiotically}
\end{itemize}

\subsection{Home-grown}
\begin{itemize}
 \item \wikilink{hp\_world\_thesis}{HP-world idea}: Catalysis based on hydrophobic interaction can 
give a rise to an efficient 
autocatalytic loop.
\item \wikilink{short\_able\_sequences}{Short sequences can have stable structure} \red{no file}: 
\wikilink{hp\_world\_thesis}{HP-world}  hypothesis heavily relies on 
assumption that relatively short sequences can have stable structure and perform function. 
Supporting literature is here. 

\end{itemize}

\section{Two polymers}
\begin{itemize}
\item \wikilink{2PMod2}{Two polymers idea} In order to start life one need two distinct types 
of polymers: informational and functional
\end{itemize}

\section{``Artificial Life''}
\begin{itemize}
 \item World Modeling
\item Movement First
\end{itemize}






  \bibliography{/data/research/31.mendeleyBibtex/library}
  \bibliographystyle{unsrt}
\end{document}
