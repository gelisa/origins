\documentclass[12pt]{paper}
\usepackage[english]{babel}
\usepackage{indentfirst}
\usepackage{graphicx}
\usepackage{latexsym}
\usepackage{amsmath}
\usepackage{amsthm}
\usepackage{amssymb,amsfonts}
\usepackage{multicol}
\usepackage{xcolor}
\usepackage{changepage}
\usepackage{hyperref}
\usepackage{fancyhdr}
\usepackage{textcomp}
\usepackage{mathtools}
\usepackage{comment}
\usepackage[normalem]{ulem}
\usepackage{marginnote}
\usepackage[numbers,sort&compress]{natbib}
\mathtoolsset{showonlyrefs=false}

\newcommand{\ga}{\alpha}
\newcommand{\gb}{\beta}
\newcommand{\gam}{\gamma}
\newcommand{\gd}{\delta}
\newcommand{\eps}{\epsilon}
\newcommand{\veps}{\varepsilon}
\newcommand{\gz}{\zeta}
\newcommand{\gt}{\theta}
\newcommand{\gi}{\iota}
\newcommand{\gk}{\kappa}
\newcommand{\gl}{\lambda}
\newcommand{\gs}{\sigma}
\newcommand{\go}{\omega}
\newcommand{\Gam}{\Gamma}
\newcommand{\gD}{\Delta}
\newcommand{\gT}{\Theta}
\newcommand{\gL}{\Lambda}
\newcommand{\gS}{\Sigma}
\newcommand{\gO}{\Omega}

%%%%%%%%%

\newcommand{\pt}[1]{\left( #1\right)}
\newcommand{\pq}[1]{\left[ #1 \right]}
\newcommand{\pg}[1]{\left\{ #1\right\}}
\newcommand{\figref}[1]{\figurename~\ref{#1}}
\newcommand{\red}[1]{\textcolor{red}{#1}}
\newcommand{\blue}[1]{\textcolor{blue}{#1}}
\newcommand{\gray}[1]{\textcolor{gray}{#1}}
\newcommand{\wikilink}[2] { \href{#1.pdf}{#2}\,(\href{#1.tex}{edit})}

\setlength{\textheight}{20cm}
\changepage{2.5cm}{3.0cm}{-4cm}{-1.0cm}{-2cm}{-2cm}{-0.6cm}{0.5cm}{0cm}
\pagestyle{fancy}
\lhead{\bf \today}
\chead{\bf Early Earth}
\rhead{EG}
\title{Early Earth -- \today}
\author{EG}
\date{\today}

\begin{document}
 \maketitle
\wikilink{home}{Home}

\wikilink{literature\_theory}{Literature: Theory}

\section{Early earth conditions from  \cite{Pace1991}}
\paragraph{When Did Life Begin?}
Earth was well formed by 4.5 billion years ago, but the time
 at which life began remains uncertain. The evidence rests
on two types of fossil record in ancient geological strata.
Hard fossils include microscopic structures that resemble
organisms in thin sectionsof rocks, and macroscopic‘stro-
matolites,” laminated mounds with structures of a type
often associated with mats of microorganisms (Schopf and
Walter, 1983). Another type of fossil record relies on the
fact that biological reactions commonly select for lighter
isotopes of carbon (reviewed by Schidlowski, 1987). There-
fore, the enrichment of 12C/‘3C in organics extracted from
ancient rocks, relative to abiogenic carbon, is some evi-
dence for the occurrence of life. The very early traces of
life are sketchy, however, because there are only a few
remnants of the early crust in which to prospect. The action
of plate tectonics has erased, by subduction, most early
strata. 
Carbon isotope fractionation studies show enrichment for C12/C13 in kerogens (carbon-containing
 polymers)
extracted from some of the oldest known rocks, about 3.8
billion years in age, from the lsua supracrustal geologic
formation (Greenland). Such evidence for life is compro-
mised, however, because thermal processes also can
cause isotopic fractionation, and the lsua rocks have been
deeply buried and heated at least once.

Putative microfossils and stromatolites  of the 3.5-
billion-year-old
 Warrawoona
 (Australia) and Swaziland
(South Africa) geological formations are generally consid-
ered more convincing as evidence for early life (reviewed
by Schopf and Walter, 1983). This evidence, however, is
also compromised.
 Based on the character of kerogen associated with putative microfossils and the morphology
of stromatolites, there is reasonable doubt as to the biolog-
ical origin of those features (Buick, 1990). The earliest
unambiguous fossil microorganisms and stromatolites are
in rocks of about 3.1 billion years in age (Mason and Von
Brunn, 1977). Our notion as to the time of life’s emergence
clearly is tenuous. Certainly it occurred before 3-3.5 billion
years ago; however, 4 billion years ago or even earlier
seems possible, depending on the compatibility of the
physical conditions at that time.

\paragraph{What Was the Character of Earth at the Tim
of the Origfn of Life?}

\begin{itemize}
 \item Surface of Earth was molten $4.2-4.5Ga$
 \item There is not much actual knowledge about that time. However theories provided some 
satisfying models (Chang et al., 1983). 
\item From model to model:
  \subitem \textopenbullet $T=500\mbox{\textcelsius}-1000$\textcelsius
  \subitem \textopenbullet pressure $500 atm\,H_20$, $40 atm\,CO/CO_2$, $10atm\,H_2$
\item The surface of Earth was reducing: $Fe^{2+}\gg Fe^{3+}$\footnote{A reducing agent (also 
called a reductant or reducer) is an element or compound that loses (or "donates") an electron to 
another chemical species in a redox chemical reaction. Since the reducing agent is losing 
electrons, it is said to have been oxidized.}
\item  Atmosphere would have been mildly oxidizing ($[CO/CO_2] \gg [CH_4]$, regardless of model.
\item High temperature of that time would prevent life from forming.  It is however unclear how 
much it was necessary to
cool in order to provide an environment suitable for the
origin of life. 
 \item  The only clue to the rate of cooling from
the geological record is that some of the earliest crustal
remnants, such as the Isua rocks, are sediments and
therefore there must have been liquid water by 3.8 billion
years ago.
\item  It seems likely that Earth’s entire surface was covered with water, but if pressure was 
high, water temperature could be $>100$\textcelsius
\item As an estimation of the upper bound of temperature at which life could occur people use 
temperatures at which modern organisms can live. Clear boundary hasn't been established yet. Some 
organisms were grown in a lab at 100\textcelsius-110\textcelsius (reviewed by Stetter et al., 
1990), and some with temperatures up to 122\textcelsius \cite{Takai2008} 
\item Surface chemistry could've been of use to help life to form:  Pyrite (Wachtershauser, 1988) 
and basaltic glasses also would have been
abundant on early Earth, and could provide surface support for complex chemistry. 
\end{itemize}

 


   \bibliography{/data/research/31.mendeleyBibtex/library}
  \bibliographystyle{unsrt} 
\end{document}
