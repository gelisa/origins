\documentclass[12pt]{paper}
\usepackage[english]{babel}
\usepackage{indentfirst}
\usepackage{graphicx}
\usepackage{latexsym}
\usepackage{amsmath}
\usepackage{amsthm}
\usepackage{amssymb,amsfonts}
\usepackage{multicol}
\usepackage{xcolor}
\usepackage{changepage}
\usepackage{hyperref}
\usepackage{fancyhdr}
\usepackage{textcomp}
\usepackage{mathtools}
\usepackage{comment}
\usepackage[normalem]{ulem}
\usepackage{marginnote}
\usepackage[numbers,sort&compress]{natbib}
\mathtoolsset{showonlyrefs=false}

\newcommand{\ga}{\alpha}
\newcommand{\gb}{\beta}
\newcommand{\gam}{\gamma}
\newcommand{\gd}{\delta}
\newcommand{\eps}{\epsilon}
\newcommand{\veps}{\varepsilon}
\newcommand{\gz}{\zeta}
\newcommand{\gt}{\theta}
\newcommand{\gi}{\iota}
\newcommand{\gk}{\kappa}
\newcommand{\gl}{\lambda}
\newcommand{\gs}{\sigma}
\newcommand{\go}{\omega}
\newcommand{\Gam}{\Gamma}
\newcommand{\gD}{\Delta}
\newcommand{\gT}{\Theta}
\newcommand{\gL}{\Lambda}
\newcommand{\gS}{\Sigma}
\newcommand{\gO}{\Omega}

%%%%%%%%%

\newcommand{\pt}[1]{\left( #1\right)}
\newcommand{\pq}[1]{\left[ #1 \right]}
\newcommand{\pg}[1]{\left\{ #1\right\}}
\newcommand{\figref}[1]{\figurename~\ref{#1}}
\newcommand{\red}[1]{\textcolor{red}{#1}}
\newcommand{\blue}[1]{\textcolor{blue}{#1}}
\newcommand{\gray}[1]{\textcolor{gray}{#1}}
\newcommand{\wikilink}[2] { \href{#1.pdf}{#2}\,(\href{#1.tex}{edit})}

\setlength{\textheight}{20cm}
\changepage{2.5cm}{3.0cm}{-4cm}{-1.0cm}{-2cm}{-2cm}{-0.6cm}{0.5cm}{0cm}
\pagestyle{fancy}
\lhead{\bf \today}
\chead{\bf HP World Model}
\rhead{EG}
\title{HP World Model -- \today}
\author{EG}
\date{\today}

\begin{document}
 \maketitle
\wikilink{home}{Home}

\wikilink{research\_calculations}{Research: Calculations}

\tableofcontents
\section{Parameters of the model}
Model is based on \cite{nowak2008prevolutionary,Ohtsuki2009,Chen2012}.
In addition to the parameters introduced in \wikilink{nowak-chen}{Nowak-Chen Model} we also have:
\begin{itemize}
\item Hydrolysis has constant rate $d_h$ per bond; it varies from $\propto 0.01$ to $\propto 1$. 
\subitem For instance, the
uncatalyzed hydrolysis of glycylglycine under neutral conditions
at 25 1C proceeds with a rate constant of $6.3  10^{-11} M^{-1} s^{-1}$
(i.e. an half life of 350 y), whereas values of $9.3 10^{-11} s^{-1}$
have been reported for glycylvaline at 37 C under neutral
conditions\cite{Smith1998}, and $1.3  10^{-10} s^{-1} $ for benzoylglycylphenylalanine
(t1/2 = 128 y)\cite{Bryant1996}.
\subitem kinetics are depending on the position of the residues within the peptide chain 
\cite{Danger2012}
\item Folding and unfolding reactions happen very quickly with the unfolding rate constants of 
$k_{unf}\gg\ga$ and folding rate constant of $k_{unf}\cdot\exp(E_{native}/kT)$.

$E_h$ in our experiments is around $1-2$kT\red{\cite{}}. $k_{unf}$ we keep $\propto 10^2$, which 
gives us range of unfolding rates from a reaction per hours and days and range of folding rates 
from a reaction per hours to fractions of a second.

\item Catalysis rate is proportional to the exponent of hydrophobic energy $E_h$ and number of 
contacting hydrophobes $n_c$: $\ga\cdot\exp(E_{h}\cdot n_{c}/kT)$. Number of hydrophobic contacts 
for the short HP-sequences is about $3-6$. With the hydrophobic energies of $1-2$kT this gives us 
catalysis rates around hours and days for one reaction.

\item Some experiments also include aggregation reactions for the long hydrophobic chains.
\end{itemize}
 






 \bibliography{/data/research/31.mendeleyBibtex/library}
  \bibliographystyle{unsrt}
\end{document}
