\documentclass[journal=jacsat,manuscript=article,layout=twocolumn]{achemso}
\usepackage[version=3]{mhchem} % Formula subscripts using \ce{}
\usepackage[T1]{fontenc}       % Use modern font encodings


\usepackage{xcolor}
\usepackage{textcomp}
\usepackage{geometry}
\usepackage[normalem]{ulem}
\usepackage{amsmath}
\usepackage{amssymb}
\usepackage{graphicx}
\usepackage{lineno,hyperref}
% \hypersetup{
%     colorlinks=true,
%      linkcolor=blue,
% %     filecolor=magenta,      
%      urlcolor=blue,
% }
% \modulolinenumbers[5]

\newcommand*{\ga}{\alpha}
\newcommand*{\gb}{\beta}
\newcommand*{\gam}{\gamma}
\newcommand*{\gd}{\delta}
\newcommand*{\eps}{\epsilon}
\newcommand*{\veps}{\varepsilon}
\newcommand*{\gz}{\zeta}
\newcommand*{\gt}{\theta}
\newcommand*{\gi}{\iota}
\newcommand*{\gk}{\kappa}
\newcommand*{\gl}{\lambda}
\newcommand*{\gs}{\sigma}
\newcommand*{\go}{\omega}
\newcommand*{\Gam}{\Gamma}
\newcommand*{\gD}{\Delta}
\newcommand*{\gT}{\Theta}
\newcommand*{\gL}{\Lambda}
\newcommand*{\gS}{\Sigma}
\newcommand*{\gO}{\Omega}
\newcommand*{\pt}[1]{\left( #1\right)}
\newcommand*{\pq}[1]{\left[ #1 \right]}
\newcommand*{\pg}[1]{\left\{ #1\right\}}
\newcommand*{\figref}[1]{\figurename~\ref{#1}}
\newcommand*{\red}[1]{\textcolor{red}{#1}}
\newcommand*{\blue}[1]{\textcolor{blue}{#1}}
% \newcommand*{\gray}[1]{\textcolor{gray}{#1}}
% \renewcommand{\baselinestretch}


%%%%%%%%%%%%%%%%%%%%%%%%%%%%%%%%%%%%%%%%%%%%%%%%%%%%%%%%%%%%%%%%%%%%%
%% Meta-data block
%% ---------------
%% Each author should be given as a separate \author command.
%%
%% Corresponding authors should have an e-mail given after the author
%% name as an \email command. Phone and fax numbers can be given
%% using \phone and \fax, respectively; this information is optional.
%%
%% The affiliation of authors is given after the authors; each
%% \affiliation command applies to all preceding authors not already
%% assigned an affiliation.
%%
%% The affiliation takes an option argument for the short name.  This
%% will typically be something like "University of Somewhere".
%%
%% The \altaffiliation macro should be used for new address, etc.
%% On the other hand, \alsoaffiliation is used on a per author basis
%% when authors are associated with multiple institutions.
%%%%%%%%%%%%%%%%%%%%%%%%%%%%%%%%%%%%%%%%%%%%%%%%%%%%%%%%%%%%%%%%%%%%%
\author{Elizaveta Guseva}
\email{elizaveta.guseva@stonybrook.edu}
\affiliation[Stony Brook University]
{Laufer Center for Physical and Quantitative Biology, Stony Brook University, Stony Brook, NY, 
(United States)}
% \altaffiliation{A shared footnote}
\author{Ronald N Zuckermann}
\affiliation{Lawrence Berkeley National Laboratory (LBNL), Berkeley, CA (United States)}
\author{Ken A Dill}
\email{dill@laufercenter.org}
\phone{+1631 632 5400}
\fax{+1631 632 5405}
\affiliation[Stony Brook University]
{Laufer Center for Physical and Quantitative Biology, Stony Brook University, Stony Brook, NY, 
(United States)}


%%%%%%%%%%%%%%%%%%%%%%%%%%%%%%%%%%%%%%%%%%%%%%%%%%%%%%%%%%%%%%%%%%%%%
%% The document title should be given as usual. Some journals require
%% a running title from the author: this should be supplied as an
%% optional argument to \title.
%%%%%%%%%%%%%%%%%%%%%%%%%%%%%%%%%%%%%%%%%%%%%%%%%%%%%%%%%%%%%%%%%%%%%
\title[]
  {Supplementary Information for ``How did prebiotic polymers become informational foldamers?''}

  



\begin{document}
\section{Simulations}
To perform our stochastic simulations, we first needed to develop appropriate simulation code 
because of the large numbers of different molecular species that must be treated here.  A 
description of the method, called the Expandable Partial Propensity Method (EPDM), and the 
corresponding C++ library, can be found at: 
https://github.com/abernatskiy/epdm~\cite{Guseva2016b}. 
The challenge is to keep track of all the molecular species and to search the full conformational 
spaces of each 
chain.  This is NP-hard.  We use the HP Sandbox algorithm\cite{lau1989lattice,Dill2008a} 
\footnote{A 
Python implementation and description can be found at: 
http://hp-lattice.readthedocs.org/en/latest/}, which is limited to maximum chain lengths of 25 
monomers.  To handle computational limitations, we restricted the total number of species to the 
level of a few thousands.  We impose this limit by introducing a dilution parameter $d$: molecules 
are randomly removed from the system with probability $\propto d$.  Physically, it represents 
molecules that diffuse out of the reaction volume.  The total numbers of molecules within the 
reaction volume vary in the range of $10^2-10^4$.  We start our simulations with a small pool of 
monomers, usually fewer than 100 molecules.  Here are the dynamical steps:

\begin{itemize}
 \item Polymerization happens when monomers react with other monomers or polymers at a rate $\beta 
= 1$:
\begin{equation}
 1\mbox{-mer}+n\mbox{-mer} \xrightarrow{\beta} (n+1)\mbox{mer}
\end{equation}


\item New monomers are imported into the system at high rate $a\gg1$. From the point of view of the 
system molecules appear out of nothing ($\emptyset$)
\begin{equation}
 \emptyset \xrightarrow{a} H\,\,\mbox{or}\,\,P
\end{equation}

\item We assume a chain can break at any internal site by hydrolysis.  This happens with rate $h$ 
per chain bond. 
\begin{equation}
 n\mbox{-mer} \xrightarrow{h} l\mbox{-mer}+(n-l)\mbox{-mer}
\end{equation}
Typical values for the half-time for the hydrolysis of a bond under neutral conditions and room 
temperature are on the order of hundreds of years\footnote{The hydrolysis rate constants of 
oligopeptides in neutral conditions are of the order of $10^{-11}-10^{-10}$: $1.3  10^{-10} 
M^{-1}s^{-1} $ for benzoylglycylphenylalanine ($t_{1/2} = 128 y$)\cite{Bryant1996}, $6.3  10^{-11} 
M^{-1} s^{-1}$($t_{1/2}=350 y$) for glycylglycine and $9.3 10^{-11}M^{-1} s^{-1}$ for glycylvaline
\cite{Smith1998}.}. Here, we explored a range of hydrolysis rates that are about $0.01-1$ of the 
polymerization rate.  Hence, our model polymerization rates are on the order of days to years.

\item We assume the system becomes diluted, at rate $d$.  This has the practical purpose of 
limiting the total population of polymer in the system.  We explored values of $d$ from $\propto 
0.01- 1\beta$.  Given the values of $a$ we used, it results in $\propto 10^2- 10^4$ chains in the 
simulation volume.
\begin{equation}
 \mbox{anything} \xrightarrow{d}\emptyset
\end{equation}

\item Folding and unfolding reactions happen much faster than the polymerization processes, with 
corresponding rate coefficients of $k_f\gg k_{u}\gg\beta$:
\begin{equation}
\begin{split}
 \mbox{folded chain}&\xrightarrow{k_u}\mbox{unfolded chain}  \\
 \mbox{unfolded chain}&\xrightarrow{k_f}\mbox{folded chain}
\end{split}
\end{equation}
 We used the most realistic values we could obtain for these rates and for the folding free 
energies for proteins.  We took $E_{nat}$ from the HP model, known folding free energies from 
experimental data~\cite{Ghosh2010,Dill2011}, and we used the relationship~\cite{Ghosh2009}:
\begin{equation}
 \ln\pt{\frac{k_f}{k_u}}=-\gD G/kT = E_{nat}/kT-N\ln z,
\end{equation} 
 where $z$ is the number of rotational degrees of freedom per peptide bond.  To account for the 
difference between the 2D model and real 3D proteins, we calibrated the parameters taken from the 
literature to yield unfolding/folding rates that are meaningful in the context of the other rates 
in our model: folding is much faster than growth and for any of the sequence in our pool 
$k_f/k_u\in (10^2,10^4)$~\cite{Ghosh2010,Dill2011} for 3D proteins.  Because the literature models 
are only mean-field, averaged over sequences, and in order to retain sequence dependence here, we 
set the unfolding rate of all sequences to the average for their lengths, and assigned all the 
sequence dependence to $k_f$.  So, we used: 
significantly.
\begin{equation}
\begin{split}
  k_u &= \exp[12-0.1 \sqrt{N} -E_H(0.5 N + 1.34)],\\
  k_f &= k_u\exp(\gD G)
\end{split}
\end{equation}
The model is not sensitive to varying these parameters over a wide range.  We use $E_h \approx 
(1-2)$kT, so $k_{unf} \approx 10^2$, which leads to a range of unfolding rates from  one unfolding 
per hour to one unfolding per day. Folding rates vary from a 
reaction per hour to a reaction per fraction of a second.

\item The catalytic step is:
\begin{equation}
\mbox{Catalyst}+H+ \underbrace{\cdots HH}_{l-1} 
\xrightarrow{\beta_{cat}}\mbox{Catalyst}+\underbrace{\cdots HHH}.
\end{equation}
 The rate enhancement is $\beta_{cat}=\beta\cdot\exp(E_{h}\cdot n_{c}/kT)$, where hydrophobic 
sticking energy is $e_H$, the number of contacting hydrophobes is $n_c$, which varies in the range 
$3-6$. With the hydrophobic energies of $e_H = 1-2$kT, this gives catalysis rates around hours to 
days per reaction. Because the EPDM supports only binary reactions, we divided the reaction above 
into to steps: interaction of catalyst with a monomer with rate $\beta$ and the reaction of this 
complex with a polymer has the rate $\beta_{cat}$. 
\end{itemize}

For each trajectory, we collected statistics only after the system reached an unchanging steady 
state.  In order 
to explore the stochasticity, we repeated every simulation for 30 times for every experiment. We 
ran all the simulations for $140s$ of internal simulation time, during which $10^6-10^9$ 
individual 
reactions had occurred. We took measurements every $10^{-6}s.$ 
For all the trajectories steady state behavior was reached no longer than $40s$ after the start of 
a simulation. Thus we considered only the last $100s$ (one million recordings) for each 
simulation. 
All the data points we used in the figures are averages over these recordings.


For all the experiments below, we used the following parameters:
\begin{enumerate}
 \item $\beta = 1$
 \item $E_h = 2kT$
 \item $z=1.2$
 
 \item $a=1000$
 \subitem Values of $a\ll 1000\,\,\mbox{or}\,\,a\gg1000$ are problematic, having numbers of 
sequences or populations either too high to calculate or too low to draw conclusions.
 
 \item $h=d=0.1$.
 \subitem When $3d\lessapprox h\leq\beta$, hydrolysis is dominates and without catalysis, there's 
an 
explosion of short sequences. 
\subitem When $3h\lessapprox d\leq\beta$, hydrolysis is unphysically small, so nothing limits the 
growth of longer sequences, even in the absence of catalysis. 
\subitem When $0.05\lessapprox d\approx h \lessapprox 0.5$ the forces of dilution and hydrolysis 
are relatively balanced and populations are neither too small or too large.
 
\end{enumerate}


\subsection*{\textit{In-silico} experiments}\label{sec:experiments}
The simulations were performed on the Laufer Center's computing cluster of CPUs. 
Source files of the models, parameters, initial conditions and random seeds can be obtained at 
\url{https://github.com/gelisa/hp_world_data}.  We performed the following computational 
experiments:


\textbf{Experiment 1: Does our bare polymerization reproduce the Flory 
distribution?}\label{sec:expt1}
We started simulations with a small pool of chains up to 3-mers. To calculate the length 
distributions, 
we calculated for each trajectory the average population of every sequence over time over all 
recordings after 40s, resulting in a million time steps.  Then we summed all the populations of a 
given 
length, obtained total populations for all $n$-mers, $n\in[1,25]$, and then computed every 
population as:
\begin{equation}
 p_n = \frac{\sum\mbox{all n-mers}}{\sum\mbox{total population}}
\end{equation}
giving probability of finding an $n$-mer of a randomly chosen molecule in the system.

The source file of the model and parameters of the simulation are located at 
\url{https://github.com/gelisa/hp_world_data/tree/master/001}

\textbf{Experiment 2. What is the effect on the distribution of just HP folding?}
We started with the same initial population as in Experiment 1. But now we introduce the 
hydrophobic 
energy $e_h= 2kT$. To calculate the result in length distribution, we computed the average 
population 
of every sequence for each trajectory over time over all the recordings after 40s, resulting in a 
million time steps. The source file of the model and parameters of the 
simulation are located at \url{https://github.com/gelisa/hp_world_data/tree/master/002}


\textbf{Experiment 3. What is the effect on the distribution of both folding and catalysis?}
In addition to folding in this \textit{in-silico} experiment, we also accounted for the pairwise 
contact interactions between two proteins, with the parameters as indicated above. We explored 
ranges of parameters.  We observed significant stability of the length distribution towards change 
of $h$ and $d$ in the range: $0.05\lessapprox d\approx h \lessapprox 0.5$.  The 
distributions we observe are quite sensitive to the choice of hydrophobic energy, as expected for 
chemical reactions, since this enters into the exponent of the rate expression. In the generally 
physical range of $e_h= 1-3 kT$, we observe a bending of the Flory distribution, as noted in the 
text. The source file of the model and parameters of the 
simulation are located at \url{https://github.com/gelisa/hp_world_data/tree/master/003}

 \bibliography{library}
 \bibliographystyle{achemso}
\end{document}
